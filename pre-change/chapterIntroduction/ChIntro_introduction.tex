

\section{Introduction to synthetic biology}

Synthetic biology aims at the rational design and construction of biological parts, devices, and systems in order to engineer organisms to perform new tasks~\autocite{Lu:2009ez, Andrianantoandro:2006bia}. A part is a basic unit, like a promoter or a ribosome binding site that when combined with other parts will make a functional unit, a device~\autocite{Heinemann:2006ht}. A device processes inputs, performs functions and produces outputs~\autocite{Andrianantoandro:2006bia}. A system comprises of a collection of devices.     

Emphasis is put on the use of engineering principles such as modularity, standardisation, use of predictive models and the separation of design and construction~\autocite{Agapakis:2009bt, Heinemann:2006ht}. A hierarchy similar to computer science is used, with cells, pathways and biochemical reactions acting as computers, modules and gates respectively~\autocite{Andrianantoandro:2006bia}. 
       
Numerous applications of synthetic biology have emerged, from altering existing metabolisms to producing synthetic drugs ~\autocite{Holtz:2010bm} or creating new synthetic life forms ~\autocite{Agapakis:2009bt}. Despite the successes there is still a lack of predictive power due to the stochasticity and lack of complete knowledge of the cellular environment ~\autocite{Andrianantoandro:2006bia}.

Synthetic biology is now entering an age where simple synthetic circuits have been built, such as toggle switches \autocite{Gardner:2000vha, Kramer:2004kq, Isaacs:2003hta, Ham:2008hh, Deans:2007cy, Friedland:2009ce}, oscillators~\autocite{Stricker:2008jqa, Fung:2005jd, Tigges:2009jx} and pulse generators~\autocite{Basu:2004gn}, but larger circuits have proven more difficult \autocite{XXX}. The leap from building low-level circuits to assembling them into complex networks has yet to be made successfully~\autocite{Lu:2009ez}, and predictable circuit behaviour remains challenging~\autocite{XXX}. Efforts to do so are plagued by intra-circuit crosstalk and incompatibility, as well as cellular noise, which can render synthetic networks non-functional \textit{in vivo} \autocite{XXX}. 


\section{Quantitative dynamical modelling in synthetic biology}

\section{Thesis Outline}

This thesis will focus on

The thesis is organised as follows:

\bfseries{Chapter 2} provides an introduction to biochemical modelling. It contains an overview of the mathematical methods that formed the basis of the methods used throughout this thesis.  

\bfseries{Chapter 3} contains a literature review on the current understanding on the dynamics of the genetic toggle switch. 


\bfseries{Chapter 4} contains 


\bfseries{Chapter 5}


\bfseries{Chapter 6}



Parts of this thesis has been published in the following articles:

\begin{enumerate}
	\item Leon, M., Woods, M., Fedorec, J. A., \& Barnes, C. P. (2016). ‘A computational method for the investigation of multistable systems and its application to genetic switches’. [Submitted].
	\item Woods, M. L., Leon, M., Perez-Carrasco, R., \& Barnes, C. P. (2016). ‘A Statistical Approach Reveals Designs for the Most Robust Stochastic Gene Oscillators.’ ACS Synthetic Biology 5(6), 459–470.
	
\end{enumerate}