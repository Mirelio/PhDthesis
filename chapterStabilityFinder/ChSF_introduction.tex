
\section{Parameter scan}
\subsection{Background}
\subsection{Methods}
\subsection{Results}

\section{Bayesian approach to model stability}
\subsection{Background}
Synthetic biology is now entering an age where simple synthetic circuits have been built, such as toggle switches \autocite{Gardner:2000vha, Kramer:2004kq, Isaacs:2003ht, Ham:2008hh, Deans:2007cya, Friedland:2009ce}, oscillators~\autocite{Stricker:2008jqa, Fung:2005jd, Tigges:2009jx} and pulse generators~\autocite{Basu:2004gn}, but larger circuits have proven more difficult \autocite{XXX}. The leap from building low-level circuits to assembling them into complex networks has yet to be made successfully~\autocite{Lu:2009ez}, and predictable circuit behaviour remains challenging \autocite{XXX}. Efforts to do so are plagued by intra-circuit crosstalk and incompatibility, as well as cellular noise, which can render synthetic networks non-functional \textit{in vivo} \autocite{XXX}. 
\par
A circuit must be robust to a fluctuating cellular environment and its response and sensitivity must be able to be fine tuned in order to orchestrate a network of circuits that function together. A robust circuit can tolerate the compound stochasticity that a chain of circuits brings, and fine tuning of its response and sensitivity enables the researcher to make it sensitive to an upstream signal as well as influence a downstream subsystem. Parts can be fine tuned by developing component libraries~\autocite{Lu:2009ez,}, but this will be of little use if the required parameter ranges for parts to make a functional complex network are unknown, and will only perpetuate the cycles of trial-and-error. A computational method to find the range of parameter values that will produce the behaviour of choice is crucial to the design process by enabling the informed selection of appropriate parts from the libraries. If it is known that gene expression must be low for a given stability, one can select a weak promoter or a low copy plasmid for the desired construct. 
\par
Both analytical and computational approaches have been deployed for the study of the toggle switch. Analytical approaches are limited to simpler models and thus require a number of assumptions to be made. The system under consideration has to be reduced to very few equations and parameters in order to make the system solvable. This requires assumptions to be made about the system that cannot always be justified, such as the quasi-steady state approximation (QSSA). The QSSA assumes that the binding/unbinding processes are much faster than any other process~\autocite{Loinger:2007vma}, thus the bound intermediate is assumed to always be in steady state. The QSSA assumption is met \textit{in vitro} but often does not hold \textit{in vivo} and its misuse can lead to large errors and incorrectly estimated parameters~\autocite{Pedersen:2007ke}. Moreover, it is generally not possible to solve even simple stochastic models analytically, and these methods are restricted to deterministic models. The computational and graph-theoretic approaches developed for the study of multistationarity generally focus on deciding on whether a given system is incapable of producing multiple steady states ~\autocite{Conradi:2007jo, Banaji:2010fh,Feliu:2013dz}. For example, ~\textcite{Feliu:2013dz} developed an approach using chemical reaction theory and generalised mass action modelling \autocite{Feliu:2013dz}. No approach exists that can handle both deterministic and stochastic systems in an integrated manner.
\par
Here we present a computational framework based on sequential Monte Carlo that takes a model and determines whether it is capable of producing a given number of (stable) steady states and the parameter space that gives rise to the behaviour. Uniquely, this can be done for both deterministic and stochastic models, and also complex models with many parameters, thus removing the need for simplifying assumptions. Our framework can be used for comparing the conclusions drawn by various modelling approaches and thus provides a way to investigate appropriate abstractions. We have made this framework into a python package, called StabilityFinder. This methodology is used to investigate genetic toggle switches and uncover the design principles behind making a bistable switch, as well as those necessary to make a tristable switch. We find that degradation rates of transcription factors are important for bistability, and that the addition of positive autoregulation can create tristable behaviour and also significantly more robust bistability when feedback strength is well balanced. Modelling transcription and translation allows us to conclude that transcriptional bursting can inhibit bistability, and also that bistability can occur even when the assumptions of time scale separation in the repressor dynamics are not met. We also examine the design principles behind the design of bistable versus tristable switches and highlight the importance of including stochastic dynamics when modelling these systems. Finally we demonstrate the ability of the framework to examine more complex systems and examine the design principles of a three gene switch. These examples demonstrate that StabilityFinder will be a valuable tool in the future design and construction of novel gene networks. 

\subsection{Methods}
%\subsubsection{StabilityFinder}

The framework is based on a statistical inference method which combines \acrfull{abc} with \acrfull{smc} \autocite{Toni:2009tr}. This simulation-based method uses an iterative process to arrive at a distribution of parameter values that can give rise to observed data or a desired system behaviour \autocite{Barnes:2011hh}.
\acrshort{abc} methods are used for inferring the posterior distribution in cases where it is too computationally expensive to evaluate the likelihood function. Instead of calculating the likelihood, \acrshort{abc} methods simulate the data and then compare the simulated and observed data through a distance function \autocite{Toni:2009tr}. Given the prior distribution $\pi(\theta)$ we can approximate the posterior distribution, $\pi(\theta\mid x)\propto f(x\mid\theta)\pi(\theta)$, where $f(x\mid\theta)$ is the likelihood of a parameter, $\theta$, given the data, $x$. There are a number of different variations of the \acrshort{abc} algorithm depending on how the the approximate posterior distribution is sampled. 
\par
The simplest \acrshort{abc} algorithm is the \acrshort{abc} rejection sampler~\autocite{Pritchard:1999td}. In this method, parameters are sampled from the prior and data simulated thorough the data generating model. For each simulated data set, a distance from that of the desired behaviour is calculated, and if greater than a threshold, $\epsilon$, the sample is rejected, otherwise it is accepted. 
\begin{algorithm}[H]
	\label{alg:ABC}
  \caption{ABC rejection algorithm}
 \begin{algorithmic}[1]
    \Statex
	\State Sample a parameter vector $\theta$ from prior $\pi(\theta)$
	\State Simulate the model given $\theta$
    \State Compare the simulated data with the desired data, using a distance function $d$ and tolerance $\epsilon$. if $d \leq \epsilon$, accept $\theta$ 
   
  \end{algorithmic}
\end{algorithm}
\noindent The main disadvantage of this method is that if the prior distribution is very different from the posterior, the acceptance rate is very low~\autocite{Toni:2009tr}. An alternative method is the \acrshort{abc} \acrfull{mcmc} developed by~\textcite{Marjoram:2003up}~\autocite{Marjoram:2003up}. The disadvantage of this method is that if it gets stuck in an area of low probability it can be very slow to converge~\autocite{Sisson:wf}. 
The method used here is based on sequential Monte Carlo, which avoids both issues faced by the rejection and MCMC methods. It propagates the prior through a series of intermediate distributions in order to arrive at an approximation of the posterior. The tolerance, $\epsilon$ for the distance of the simulated data to the desired data is made smaller at each iteration. When $\epsilon$ is sufficiently small, the result will approximate the posterior distribution~\autocite{Toni:2009tr}.  
\par
To investigate the multistable behaviour of systems, a number of extensions to existing approaches are required. For a given set of parameter values, sample points are taken across initial conditions using latin hypercube sampling \autocite{XXX}, and the ensemble system simulated in time until steady state. The distance function in \acrshort{abc} is replaced by a distance on the desired stability of the simulated model. To do this we cluster the steady state coordinates using k means clustering and use the gap statistic to determine the number of clusters \autocite{XXX}. The algorithm is summarised below.
\begin{algorithm}[H]
\label{alg:StabilityFinder}
\caption{StabilityFinder algorithm}
 \begin{algorithmic}[1]
    \Statex
	\State Initialise $\epsilon$ 
	\Let{population p}{1}
	\If{p $= 1$}
		\State Sample particles ($\theta$ ) from priors
		\Else
			\State Sample particles from previous population
			\State Perturb each particle by $\pm$ half the range of the previous population (j) to obtain new perturbed population (i).
	\EndIf
	\State Sample initial conditions via Latin Hypercube Sampling.
    \State Simulate each particle to obtain steady state values.
    \State Cluster steady state
	\State Reject particles if d $\textgreater$ $\epsilon$.
    \State Calculate weight for each accepted $\theta$
	%\State $w_{t}^{(i)} = \begin{cases} 1, & \mbox{if } p = 0 \\\frac{\pi(\theta_{t}^{(i)})}{\sum_{j=1}^N w_{t-1}^{(j)} K_{t}(\theta_{t-1}^{(j)}, \theta_{t}^{(i)})}, and \mbox{if } p \geq  0. \end{cases}$
	\State Normalise weights
	\Repeat{steps 3 - 15} \Until{$\epsilon \leq \epsilon_T$}
  \end{algorithmic} 
\end{algorithm}
\noindent This algorithm is available as a Python package, called StabilityFinder. %which is available at \url{https://github.com/ucl-cssb/StabilityFinder.git}.  
The user provides an SBML model file \autocite{XXX} and an input file that contains all the necessary information to run the algorithm, including the desired stability and the final tolerance, $\epsilon$, for the distance from the desired behaviour necessary for the algorithm to terminate. The flow of execution is illustrated in Figure~\ref{fig:fig1}. Since the algorithm is computationally intensive, all deterministic and stochastic simulations are performed using algorithms implemented on \acrfull{gpu}s \autocite{XXX}.

\subsection{Testing StabilityFinder}

\textcite{Gardner:2000vha} constructed the first synthetic genetic toggle switch~\autocite{Gardner:2000vha}. Their model consisted of two mutually repressing transcription factors, as shown in Figure~\ref{fig:fig2}A, and in the deterministic case is defined by the following \acrshort{ode}s.
\begin{align*}
\frac{du}{dt} &= \frac{a_1}{1+v^{\beta}} - u\\
\frac{dv}{dt} &= \frac{a_2}{1+u^{\gamma }} - v,
\end{align*}
where \textit{u} is the concentration of repressor 1, \textit{v} the concentration of repressor 2, $a_1$ and $a_2$ denote the effective rates of synthesis of repressors 1 and 2 respectively,  \textit{$\beta$} is the cooperativity of repression of promoter 1 and \textit{$\gamma$} of repressor 2. Gardner \textit{et al} studied the deterministic case and concluded that there are two conditions for bistability for this model; that $a_1$ and $a_2$ are balanced and that $\beta$, $\gamma$ \textgreater 1 \autocite{Gardner:2000vha}. In order to test StabilityFinder we used it to find the posterior distribution for which this model exhibits bistable behaviour. We therefore set the desired behaviour to two (stable) steady states, and using a wide range of values as priors as shown in the Supplementary Information, we used StabilityFinder to find the parameter values necessary for bistability to occur. The posterior distribution calculated by StabilityFinder for the Gardner deterministic case is shown in Figure~\ref{fig:fig2}B.
\par
These results agree with the results reported by~\textcite{Gardner:2000vha}~\autocite{Gardner:2000vha}. For this switch to be bistable $a_1$ and $a_2$ must be balanced while $\beta$ and $\gamma$ must both be \textgreater 1, as can be seen in the marginal distributions of $\beta$ and $\gamma$ in Figure~\ref{fig:fig2}B. The conditions set by the original paper for parameters $a_1$ and $a_2$ are met, as the joint distribution shown in Figure~\ref{fig:fig2}B matches the bifurcation lines calculated in the original paper {\color{red} How do you know, did you calculate these? I think you should try and plot them on the posterior}. 
We next applied StabilityFinder to the case of the Gardner switch under stochastic dynamics using the same priors as the deterministic case, and again searched the parameter space for bistable behaviour. The posterior is shown in Figure~\ref{fig:fig2}C. We can see that the conditions on the parameters required for bistability in the deterministic case generally still stand in the stochastic case. There appears to be slightly looser requirements on the parameters of the stochastic model (wider marginal distributions), which is expected due to the nature of clustering deterministic steady states versus stochastic steady states. The gap statistic is used in the case of the stochastic case, as it is capable of dealing with noisier data whereas a simpler and faster algorithm is used for clustering the deterministic solutions. The study on the effect of the clustering methods on robustness is shown in the Supplementary Information. These results demonstrate that StabilityFinder can be used to find the parameter values that can produce a desired stability and allow us to confidently apply the methodology to more complex models.

\begin{table}[h]
\centering
\caption{Gardner switch priors in the deterministic and stochastic cases}
\label{tab:gard_det_stoch}
\begin{tabular}{@{}cccccc@{}}
\toprule
\multicolumn{4}{c}{Parameters}                          & \multicolumn{2}{c}{Species} \\ \cmidrule(lr){1-4}
\cmidrule(lr){5-6}
$a_1$ & $\beta$ & $a_2$ & \multicolumn{1}{c}{$\gamma$} & $s_1$        & $s_2$        \\
0-60  & 0-5     & 0-60  & 0-5                           & 0-100        & 0-100        \\ \bottomrule
\end{tabular}
\end{table}


\subsection{Calculating robustness}

We calculate robustness $R$ by using a Monte Carlo sampling rejection algorithm by taking random samples from the prior and accepting those found in the functional region $F$. Robustness is then defined as the number of accepted samples divided by the total number of samples.

 In order to test this algorithm we use a cube of equal length sides as the prior. Instead of a posterior distribution we draw random samples from uniform $u(a, b)$ in three dimensions or multivariate normal $N(\mu, \sigma)$ distribution. In the case of the uniform distribution, we adjust the range of the data by drawing samples from a range incrementally smaller than the prior cube, always centred in the centre of the prior cube. In the case of a normal distribution $\mu$ is set to the centre of the cube and we vary $\sigma$ to get a distribution with increasingly larger range, approaching the full range of the prior.  

 We calculate the true robustness $R_t$ by calculating the ratio of the volume of the data to the volume of the prior cube. In the case where data$~u(a, b)$, the data is treated as a cube with length $a = (b - a)$ and volume $V = a^3$. In the case where data$~N(\mu, \sigma)$, the data is treated as a sphere with radius $r = \frac{(max - min)}{2}$ and volume $V = \frac{4}{3} * \pi * r^2$. 
 
\begin{algorithm}[H]
	\label{alg:robustness}
  \caption{Calculating robustness via Monte Carlo sampling rejection}
 \begin{algorithmic}[1]
    \Statex
    \Let{$R$}{Robustness}
    \Let{$F$}{Functional region}
    \For{Each variance sample}
    	\State Generate data set
    	\State Calculate true $R_t$: 
    	\Let{$V_p$}{Prior Volume}
    	\Let{$a_p$}{Prior max - min} \
    	\State $V_p = a_p^3$ 
    	
    	\If{data$~N(\mu, \sigma)$}
    		\State Get min and max of data \
    		\Let{r}{$\frac{max - min}{2}$} \
    		\State $V_d = \frac{4}{3} * \pi * r^2$ 
    	\EndIf
    	
    	\If{data$~u(min, max)$}
    		\Let{$a$}{$max - min$} \
    		\State $V_d = a^3$
    		\State $R_t = \frac{V_d}{V_p}$
    	\EndIf
    	
    	\State Define grid
    	\State Do \acrfull{kde} on data
    	\For{Each repeat}
    		\State Sample from priors 
    		\Let{$Z$}{5th percentile of KDE}
    		\Let{$F$}{KDE > $Z$}
    		\If{sample $\in$ grid}
    		\State Find sample's position in grid
				\If{position $\in$ $F$}
    				\State $accepted += 1$
    			\EndIf
    		\EndIf
    		\State $R$ = $\frac{accep
    	ted}{number of samples}$
    	\EndFor

  \EndFor  
   \State \Return mean($R$), standard deviation($R$), $R_t$
  \end{algorithmic}
\end{algorithm}	

\clearpage


\subsubsection{Normal distribution}


\begin{figure}[htbp]
\centering
\includegraphics[scale=0.6]{chapterStabilityFinder/images/robustness_calc_norm.png}
\caption[LoF caption]{}
\label{}
\end{figure}
\clearpage
\paragraph{Studying the effect of correlation}
\begin{figure}[htbp]
\centering
\includegraphics[scale=0.6]{chapterStabilityFinder/images/normal_correl.png}
\caption[LoF caption]{}
\label{}
\end{figure}
\clearpage
\paragraph{Comparing increasing dimensions}
\begin{figure}[htbp]
\centering
\includegraphics[scale=0.6]{chapterStabilityFinder/images/robustness_s_normal_dimens.pdf}
\caption[LoF caption]{}
\label{}
\end{figure}
\clearpage
\subsubsection{Uniform distribution}
\begin{figure}[htbp]
\centering
\includegraphics[scale=0.6]{chapterStabilityFinder/images/robustness_calc_unif.png}
\caption[LoF caption]{}
\label{}
\end{figure}
\clearpage
\paragraph{Comparing increasing dimensions}
\begin{figure}[htbp]
\centering
\includegraphics[scale=0.6]{chapterStabilityFinder/images/robustness_s_uniform_dimens.pdf}
\caption[LoF caption]{}
\label{}
\end{figure}
\clearpage


\subsection{Lu toggle switch models}

We next analyzed an extension of the Gardner switch model developed by~\textcite{Lu:2014kc}~\autocite{Lu:2014kc}. They considered two types of switches, the classic switch consisting of two mutually repressing transcription factors (model \acrshort{cs-lu}), as well as a \acrfull{dp-lu}.  The classical switch was found to be bistable given the set of parameters used, while the DP switch was found to be tristable ~\autocite{Lu:2014kc}. The classical model used in their study is given by the following system of \acrshort{ode}s.
\begin{align*}
\dot{x} & = g_{x}\, H^{S}_{xy}(y) -k_{x}x \\
\dot{y} & = g_{y}\,H^{S}_{yx}(x) -k_{y}y,
\end{align*}
where
\begin{align*}
H^{S}_{I}(x) &= H^{-}_{I}(x)+\lambda_{I}H^{+}_{I}(x) \\
H^{-}_{I}(x) &= 1 \big/\left[1+(x/x_{I})^{n_{I}}\right] \\
H^{+}_{I}(x) &= 1-H^{-}_{I}(x),
\end{align*}
and $g_I$ represents the production rate, $k_I$ the degradation rate, $n_I$ the Hill coefficient, $x_I$ the Hill threshold concentration and $\lambda_I$ the fold change of the transcription rates, and $I\in\{xy, yx, xx, yy\}$. 
\par
For the parameter values used in the Lu study, the classical switch exhibits three steady states (Figure~\ref{fig:fig3}), two of which are stable and one is unstable. Using StabilityFinder with priors centred around the parameter values used in the original paper (see Supplementary Information), we can identify the most important parameters for achieving bistability. The posterior distribution of these models are shown in Figure~\ref{fig:fig3}A. We find that the parameters representing the rates of degradation of the transcription factors in the system ($k_x$,$k_y$) must both be large in relation to the prior ranges for bistability to occur. Protein degradation rates have been shown to be important for many system behaviours including oscillations \autocite{XXX}. {\color{red} Anything else regarding this classical model? with more parameters}

It is known that that the addition of positive autoregulation to the classical toggle switch can induce tristability  \autocite{XXX, Lu:2014kc}. Here we investigate the interplay of positive autoregulation on the values of the other parameters in the model. We extended the analysis presented in ~\textcite{Lu:2014kc} by including the switch with single positive autoregulation (model \acrshort{sp-lu}), where an asymmetry of positive feedbacks is present between the two genes. The advantage of using StabilityFinder over solving the system analytically is that the full parameter space is explored rather than solving the system for a single set of parameters. This allows us to deduce model properties that could not otherwise be identified. Robustness to parameter fluctuations can be explored, as well as parameter correlations and restrictions on the values they can take while still producing the desired behaviour.  
\par
The \acrshort{dp-lu} model is given by
\begin{align*}
\dot{x} = f_{x}(x,y) &= g_{x}\, H^{S}_{xy}(y)\, H^{S}_{xx}\,(x)-k_{x}x \\
\dot{y} = f_{y}(x,y) &= g_{y}\,H^{S}_{yx}(x)\,H^{S}_{yy}\,(y)-k_{y}y,
\end{align*}
whereas the \acrshort{cs-lu} switch is modelled using the following \acrshort{ode} system
\begin{align*}
\dot{x} & = g_{x}\, H^{S}_{xy}(y)\, H^{S}_{xx}\,(x)-k_{x}x \\
\dot{y} & = g_{y}\,H^{S}_{yx}(x) - k_{y}y.
\end{align*}
We find that the switch with single positive autoregulation is capable of bistable behaviour as seen in Figure~\ref{fig:fig3}B, but this is only possible when the strength of the promoter under positive autoregulation, $gx$, is small. There appear to be no such constraints on the strength of the original, unmodified, promoter, $gy$.  
\par
Upon examination of the \acrshort{dp-lu} model, we also find that tristability in the switch is relatively robust, as tristability is found across a large range of parameter values, with no parameters strongly constrained {\color{red} Can we say a bit more here?}. Two types of tristable behaviour are identified, one where the third steady state is at (0,0) and and one where the third steady state has non-zero values. This result agrees with previous work by~\textcite{Guantes:2008gs}, who found that a switch can exhibit two kinds of tristability, one in which the third steady state is high (\RNum{3}$_H$) and one in which it is low (\RNum{3}$_L$)~\autocite{Guantes:2008gs}. %The difference between the two tristable switches is explored in a later section. 

deterministic dynamics~\autocite{Guantes:2008gs}. The classical switch is also capable of both bistable and tristable behaviour when stochastic dynamics capture small number effects~\autocite{Ma:2012dt}. It is therefore of great interest to understand the conditions under which these two behaviours occur in both stochastic and deterministic scenarios. 
\par
In order to do this using StabilityFinder, we first obtained posterior distributions for bistable and tristable behaviours in the deterministic case (\acrshort{dp-lu} model) and then compared the individual parameter distributions (Figure~\ref{fig:fig6} and Supplementary Information). From analysis of the one dimensional marginal distributions it appears that there is no difference in the parameter values that allow a switch to be bistable versus tristable (Figure~\ref{fig:fig6}B). However, upon examination of the two dimensional marginal distributions we find that the correlation structure of a small subset of the posterior parameter values is in fact different under the two behaviours (Figure~\ref{fig:fig6}C). Most notably, we find differences in the bivariate distribution of the two parameters for gene expression, $gx$ and $gy$, as highlighted in Figure~\ref{fig:fig6}C, box 1. In the tristable case the distribution is more constrained than in the bistable case, as both parameters must be small for tristability to arise. 
Parameters $xYY$ and $gy$ are tightly constrained in the tristable case and both required to be small, but less so in the bistable case (Figure~\ref{fig:fig6}C, box 3). Another notable difference is between parameters $xXX$ and $lXX$ shown in Figure~\ref{fig:fig6}C, box 4, where they are constrained in the bistable case but not the tristable case. Interestingly, we also find parameter correlations conserved between the bistable and tristable case, as seen in Figure~\ref{fig:fig6}C, box 2, where parameters $lXX$ and $gx$, positive autoregulation and gene expression are negatively correlated in both cases.  This highlights the importance of treating unknown parameters as distributions rather than fixed values when studying the parameter values of a model, as they are capable of uncovering not only the ranges and values needed but also the correlations between parameters that would not have otherwise been detected.

To further demonstrate the flexibility of our framework we investigated a system capable of higher stabilities. Multistability is found in differentiating pathways, like the myeloid differentiation pathway~\autocite{Ghaffarizadeh:2014bt, Cinquin:2005go}. We allow for these more complex dynamics by extending the Lu DP model by adding another gene, making it a three gene switch.  This new system is depicted in Figure~\ref{fig:fig8}A. In StabilityFinder we look for six steady states, the output being in nodes $X$ and $Y$ and using the priors shown in Table~\ref{tab:multi_priors}. We successfully find that the system is capable of six steady states, as shown in Figure~\ref{fig:fig8}C and as predicted mathematically and shown in Figure~\ref{fig:fig8}B. Interestingly, and consistent with the results presented above, we find that the most constrained parameters for this behaviour are again the degradation rate of the proteins, $kx$. Additionally we find that the Hill coefficients for the repressors, $nxy$, are tightly contrained to be smaller than 1.5 as seen in Figure~\ref{fig:fig8}D. {\color{red} What else can we do here? Can we simulate the system switching or something?} This example demonstrates that StabilityFinder can be used to elucidate the dynamics of more complex network architectures, which will be key to the successful design and construction of novel gene networks as synthetic biology advances.

\begin{table}[htpb]
\centering
\caption{Priors of the classical(\acrshort{cs-lu}), single positive (\acrshort{sp-lu}) and double positive (\acrshort{dp-lu}) models.}
\label{tab:lu_all}
\begin{tabular}{@{}ccccc@{}}
\toprule
Parameter                                            & Symbol & \acrshort{cs-lu}       & \acrshort{sp-lu}        & \acrshort{dp-lu}       \\ \midrule
\multirow{2}{*}{Production rate}                & gx        & 30-50   & 1-2       & 1-100    \\
                                                & gy        & 30-50   & 20-25     & 1-100    \\[4pt]
\multirow{2}{*}{Degradation rate}               & kx        & 0-0.5   & 50-55     & 0-1      \\
                                                & ky        & 0-0.5   & 48-52     & 0-1      \\[4pt]
\multirow{2}{*}{Hill coefficient}               & nxy       & 1-5     & 30-35     & 0-10     \\
                                                & nyx       & 1-5     & 0.1-0.2   & 0-10     \\[4pt]
\multirow{2}{*}{Hill thresholds concentration}  & xxy       & 100-300 & 2-3       & 100-1000 \\
                                                & xyx       & 100-300 & 0.4-0.6   & 100-1000 \\[4pt]
\multirow{2}{*}{Transcription rate fold change} & lxy       & 0-0.5   & 0.02-0.04 & 0-1      \\
                                                & lyx       & 0-0.5   & 0.02-0.04 & 0-0.2    \\[4pt]
\multirow{2}{*}{Hill coefficient}               & nXX       & -       & 25-30     & 0-10     \\
                                                & nYY       & -       & 0.01-0.02 & 0-10     \\[4pt]
\multirow{2}{*}{Hill thresholds concentration}  & xXX       & -       & 0.4-0.5   & 50-500   \\
                                                & xYY       & -       & 1-3       & 50-500   \\[4pt]
\multirow{2}{*}{Transcription rate fold change} & lXX       & -       & 65-72     & 1-20     \\
                                                & lYY       & -       & 0.02-0.04 & 1-20     \\ \hline
\end{tabular}
\end{table}


\begin{table}[htpb]
\centering
\caption{Priors used in the three-node switch}
\label{tab:multi_priors}
\begin{tabular}{@{}ccc@{}}
\toprule
Parameter                           & Symbol & Range   \\ \midrule
Production rate                & gx        & 3-5    \\
Degradation rate               & kx        & 0-0.2     \\
Hill coefficient               & nxy       & 0-2    \\
Hill thresholds concentration  & xxy       & 140-160 \\
Transcription rate fold change & lxy       & 0-0.2     \\
Hill coefficient               & nxx       & 2-4     \\
Hill thresholds concentration  & xxx       & 90-110  \\
Transcription rate fold change & lxx       & 8-12    \\ \bottomrule
\end{tabular}
\end{table}


\subsection{Mass Action switches}						
Although the DP switch can achieve tristability, we investigated how the addition of positive autoregulation alters the robustness of the switch for bistable behaviour. Here we define a robust system as a device that can withstand fluctuations in parameter values and still produce the desired behaviour (parametric robustness). Feedback loops are well known key regulatory motifs~\autocite{Brandman:2005ci}. Although negative feedback loops are essential for homeostasis and buffering~\autocite{Thomas:1995id} and can increase robustness to extrinsic noise sources, we focussed on the addition of positive feedback because this has been shown to increase robustness in other systems \autocite{XXX}. We use StabilityFinder to compare the two models, and determine whether its worthwhile building a bistable circuit with added positive feedback.
%StabilityFinder allows us to study more complex systems and to take stochasticity into account. This is relevant in a biological systems in which realistic models tend to involve many species and interactions and where small molecule numbers and external noise are non negligible. 
\par
In order to study the switch system in the most realistic way, we avoid using the quasi-steady state approximation (QSSA) that is often used in modelling the toggle switch. Using mass action, this changes the two-equation system used in~\textcite{Lu:2014kc}~\autocite{Lu:2014kc} into a system of 18 equations and 7 parameters in the classical switch case. These are shown in the Supplementary Information.% Using mass action, the two models representing the  C switch with no autoregulation and DP switch are shown in Figure~\ref{fig:fig4} {\color{red} No they're not}.  

Given the posterior distributions shown in Figure~\ref{fig:fig4}, we find that the parameter for transcription factor degradation ($deg$) has to be high, a result that reflects the findings for the Lu models above {\color{red} Do we really see that??}. We also find that the addition of positive feedback loops greatly increases the system's robustness to parameter fluctuations as seen in Figure~\ref{fig:fig4}D. Adding positive feedback loops to the model allows it to be bistable over a greater range of parameter values, but only when the system is symmetric. When this constraint is removed we no longer see a difference in robustness between the CS-MA and DP-MA models. {\color{red} What does this mean? }

%This indicates that small fluctuations in parameters in the cellular environment will not affect the system's ability to be bistable, and thus makes it more suitable for use in synthetic biological applications where a very constrained parameter set can be too restrictive.% This makes it a better candidate for building new synthetic devices based on the toggle switch design.

%We identified the parameter region within which these models are bistable, information that is important when building such a device in the lab. In the future, by selecting the system components accordingly, the parameter values can be adjusted \textit{in vivo}. For example, the desired level of gene expression can be accomplished by selecting the appropriate RBS sequence~\autocite{Salis:2009gk}. Another method to modify the parameter values \textit{in vivo} is to select the promoter to have the strength corresponding to the levels of gene expression and repression desired. Activity of each promoter can be measured and standardised~\autocite{Kelly:2009bj} making this process possible. For a system requiring more than one promoter, these can be efficiently selected from a promoter library using a genetic algorithm~\autocite{Wu:2011bq}. These standardised interchangeable components with known sequence and activity~\autocite{Kelly:2009bj,Canton:2008fv} can be selected and used to construct a desired system and replicate the parameter values found using StabilityFinder.

We find that in the stochastic case, both the simple switch, S-MA, and positive autoregulation switch, DP-MA, are capable of both bistable and tristable behaviour. The fact that tristability can occur in the classical model is consistent with the effect of small molecule numbers; if gene expression remains low, it provides the opportunity for small number effects to be observed, and the third steady state to stabilise \autocite{Ma:2012dt}. In order to ensure that the tristable switches found in the stochastic case are truly tristable, we re-sample the posterior distributions and simulate to steady state. If the resulting phase plots are tristable then we know that the posterior truly represents tristability. As can be seen in Figure~\ref{fig:fig7}B, differences in the parameter values are observed between the bistable and tristable switches, in both CS-MA and DP-MA models. The design principles for both the CS-MA model and the DP-MA model are summarised in Table~\ref{tab:des_prin}

\begin{table}[]
\centering
\caption{Design principles of bistable and tristable switches}
\label{tab:des_prin}
\begin{tabular}{@{}ccccc@{}}
\toprule
                    & \multicolumn{2}{c}{\textbf{CS-MA}} & \multicolumn{2}{c}{\textbf{DP-MA}} \\ \cmidrule(l){2-3}\cmidrule(l){4-5}
                    & \textit{Bistable}    & \textit{Tristable}   & \textit{Bistable}    & \textit{Tristable}   \\\cmidrule(l){2-2}\cmidrule(l){3-3}\cmidrule(l){4-4}\cmidrule(l){5-5}
dimerisation        & High        & Low         & High        & Low         \\
protein degradation & -           & -           & -           & Low         \\
dimer degradation   & Low         & -           & Low         & -           \\\bottomrule
\end{tabular}
\end{table}

\section{Conclusions}
We have developed an algorithm that can identify the parameter regions necessary for a model to achieve a given number of stable steady states. The novelty in our framework over existing methodology is that complex models can be analyzed assuming both deterministic and stochastic dynamics. We have shown that the algorithm can be used to infer parameter ranges and compare robustness between models. We found that adding positive feedback increases the robustness of a toggle switch model to parameter fluctuations. We also found that stochastic modelling can enable a model to achieve stabilities not possible in the deterministic case. We uncovered the design principles that make a tristable switch and finally we found that a three-node switch is capable of hexastability.
\par
Tools that can identify parameter regions that give rise to specific behaviours will be key for the success of synthetic biology. In the future, by selecting the system components accordingly, the parameter values can be adjusted \textit{in vivo}. For example, the desired level of gene expression can be accomplished by selecting the appropriate RBS sequence~\autocite{Salis:2009gk}. Another method to modify the parameter values \textit{in vivo} is to select the promoter to have the strength corresponding to the levels of gene expression and repression desired. Activity of each promoter can be measured and standardised~\autocite{Kelly:2009bj} making this process possible. For a system requiring more than one promoter, these can be efficiently selected from a promoter library using a genetic algorithm~\autocite{Wu:2011bq}. These standardised interchangeable components with known sequence and activity~\autocite{Kelly:2009bj,Canton:2008fv} can be selected and used to construct a desired system and replicate the parameter values found using StabilityFinder.
\par
The methodology presented here can also be used to study the topology of more complex multistable switches that exist in natural biological systems such as developmental pathways. We also limited our framework to the objective behaviour of a given number of stable steady states. This could be extended to examine systems with a given switching rate or systems robust to a particular set of perturbations, both of which could be of great importance for building more complex genetic circuits.
\par
More generally our results highlight that changing the level of abstraction included in the model can significantly alter the observed multistationarity. This is further exacerbated by a change in the dynamics from deterministic to stochastic. These results for the case of multistationarity, assuming they translate to systemic behaviour more generally, suggest that a programme of experimental work, combined with systems modelling, is required to understand the rules of thumb for abstraction in model based design of synthetic biological systems.   

\begin{figure*}[h]
\begin{center}
\includegraphics[scale=0.9]{chapterStabilityFinder/images/figure-01.png}
\caption[LoF caption]{\label{fig:fig1}: Using sequential Monte Carlo to examine system stability. The algorithm takes as input a model (A) and evolves it to the stability of choice (C) via intermediate populations. In this example model shown in A, There are two species and two parameters. For the model to be bistable, the phase plot of the two species of interest must have two distinct densities, as shown in (C). The parameter space of the model is searched through our algorithm until the resulting simulations give rise to bistability. The parameter values for the model that demonstrated the desired behaviour are given as an output (D). The output consists of the accepted values for each parameter, as well as each density plotted against the other. This allows us to uncover correlations between parameter values. We made this algorithm into a python package, called StabilityFinder. The overview of the algorithm is shown in (E).}
\end{center}
\end{figure*}
\clearpage

\begin{figure*}[h]
\begin{center}
\includegraphics[width=\textwidth]{chapterStabilityFinder/images/figure-02.png}
\caption[LoF caption]{ \label{fig:fig2}: Elucidating the stability of the Gardner switch. The Gardner model (A) consists of two mutually repressing transcription factors. It can be reduced to a two-equation system, where $u$ and $v$ are the two transcription factors, $a_1$,$a_2$ are their effective rates of synthesis, $u$,$v$ are their concentrations and $\beta$, $\gamma$ represent the cooperativity of each promoter. There are four parameters in the model, for which we want to find the values for which this system is bistable. We use StabilityFinder to find the posterior distribution of the bistable Gardner switch, deterministically (B) and stochastically (C). The posterior distributions are shown as the density plots of each parameter as well as each one plotted against the other. }
\end{center}
\end{figure*}
\clearpage

\begin{figure*}[p]
\begin{center}
\includegraphics[width=\textwidth]{chapterStabilityFinder/images/figure-03.png}
\caption[LoF caption]{ \label{fig:fig3}: The three variants of the Lu models. (A)\textbf{CS (deterministic).}  The classic switch with no autoregulation is bistable. The most restricted parameters for this behaviour are $kx$ and $ky$ which both have to be high relative to the prior while the net protein production for $X$ and $Y$ myst be balanced. (B)\textbf{SP (deterministic).}The extended Lu model with a single positive autoregulation on $X$. This model is bistable when $gx$ is small. (C) \textbf{DP (deterministic).} The Lu model with double positive autoregulaiton is tristable, and its posterior distribution shown here. We find two types of tristable behaviour, one where the third steady state is zero-zero and one where the third state is high (non-dead).}
\end{center}
\end{figure*}
\clearpage

\begin{figure*}[h]
\begin{center}
\includegraphics[scale=0.5]{chapterStabilityFinder/images/figure-04.png}
\caption[LoF caption]{ \label{fig:fig4}: Adding positive feedback increases robustness to parameter fluctuations. We compare the posterior distributions of the simple mass action switch and the switch with added double positive autoregulation (A). The two models were both capable of bistable behaviour (B). The two posterior distributions are shown in (C). Comparing the functional region $F$ of the posterior distributions (D) we find that there is a five-fold increase in robustness when positive autoregulation is added. This is not the case for the asymmetric switch in which there is no significant difference in robustness between the CS-MA and DP-MA models.}
\end{center}
\end{figure*}
\clearpage


\begin{figure*}[h]
\begin{center}
\includegraphics[scale=0.6]{chapterStabilityFinder/images/figure-05.png}
\caption[LoF caption]{ \label{fig:fig5}: (A) Separating transcription and translation in the mass action model. (B) Transcriptional bursting does not allow for a bistable switch. (C) The Quassi Steady state approximation assumption that the dimerization reaction ($dim$) is much faster than its reverse ($dim\_r$) cannot be justified here. The median of the data (red) lies below the x=1 line (black dashed line) which indicates that in the majority of the particles in the posterior $\frac{dim}{dim\_r} < 1$  (D) No difference was found in the robustness to parameter fluctuations for the two models tested here.}
\end{center}
\end{figure*}

\begin{figure*}[h]
\begin{center}
\includegraphics[width=\textwidth]{chapterStabilityFinder/images/figure-06-alt.png}
\caption[LoF caption]{ \label{fig:fig6}: Design principles of a tristable switch. (A) Using the Lu model with added positive autoregulation we uncover the design principles dictating if a switch will be bistable or tristable. (B) By considering each parameter separately we cannot find a significant difference in the parameter values acceptable for a bistable versus a tristable switch. (C) By considering the bivariate distributions of the parameters we can uncover the differences in the parameters of a bistable switch compared to a tristable switch. The posterior distribution of the bistable switch is shown in red and of the tristable switch in blue. The bivariate distributions for which a difference is observed between a bistable and a tristable switch are in black boxes. }
\end{center}
\end{figure*}


\begin{figure*}[h]
\begin{center}
\includegraphics[scale=0.6]{chapterStabilityFinder/images/figure-07.png}
\caption[LoF caption]{ \label{fig:fig7}: Tristability is possible in the mass action toggle switch models only when simulated stochastically. (A) the simple toggle switch with no autoregulation can be both bistable and tristable. The two posteriors are shown, where the posterior distribution of the bistable switch is shown in red and of the tristable switch in blue. From the posterior distribution we can deduce the the dimerization parameter must be small for tristability to occur but large for bistability. The switch with double positive autoregulation and its posterior distributions for the bistable and tristable case are shown in (B). A sample phase plot of a stochastic tristable and bistable mass action switch is shown in (C). Comparing the robustness of the switch with and without positive autoregulation, we don't find a significant difference (D).}
\end{center}
\end{figure*}
\clearpage

\begin{figure*}[h]
\begin{center}
\includegraphics[scale=0.7]{chapterStabilityFinder/images/figure-08.png}
\caption[LoF caption]{ \label{fig:fig8}: The three-node mutual repression model, with added positive auto-regulation on each node. (A) The model. The model is studied in two dimensions using StabilityFinder, $X$ and $Y$. (B) The phase plot of the resulting final population of StabilityFinder. There are 6 steady states. (C) The posterior distribution of the 6-steady state three-node system. Parameters $kx$ and $nxy$ are the most constrained.}
\end{center}
\end{figure*}

