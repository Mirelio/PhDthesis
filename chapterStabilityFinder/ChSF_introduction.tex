\section{Introduction}

A challenge that synthetic biology is facing is the ability to build synthetic devices that are robust to changing cellular contexts. Unknown initial conditions and parameter values as well as the variability of the cellular environment, extracellular noise and crosstalk makes the majority of synthetic genetic devices non-functional \autocite{Chen:2009ea}. There has been great progress in the  quantity and quality of devices being created(), but we are still lagging behind in our ability to rationally design a device and minimize the time-consuming and expensive experimental trial and error. 

One of the most common devices used StabilityFinder in synthetic biology is the genetic toggle switch. A toggle switch consists of a set of transcription factors that mutually repress each other \autocite{Gardner:2000vha}. Toggle switches play a major role in binary cell fate decisions like stem cell differentiation, as they are capable of exhibiting bistable behaviour. Bistability is defined as the system being able to have two distinct phenotypic states but no intermediate state. Bistability is a property that is important in nature and a valuable resource to tap into in synthetic biology. It allows cells to alter their response to environmental cues and increases the overall population fitness by 'hedge-betting' the response of the population.

Bistability ensures that the differentiating cell will go down one pathway, or the other, with no intermediate phenotypes being possible. This is vital for the correct development of a cell in a specific pathway. One example is the trophectoderm differentiation pathway, in which a mutually inhibitory toggle switch exists between Oct3/4 and Cdx2. This determines whether an Embryonic Stem cell will differentiate into a Trophectoderm cell, if Cdx2 dominates the system, or an Inner Cell Mass cell if Oct3/4 dominates \autocite{Niwa:2005fz}. Bistability is critical in this system as a cell must differentiate into either a trophectoderm cell or an inner cell mass cell, thus the signal to do so must be straightforward. In the case of the GATA1 and PU.1 toggle switch, the transcription factor pair controls the fate of the common myeloid progenitors, and the two possible differentiation paths are erythroid and myeloid blood cells \autocite{Chickarmane:2009by}. The double-negative feedback loop created by the mutually repressive pair of transcription factors sustains the system in balance until an external stimulus causes one of the two transcription factors to increase in concentration. The increased concentration of one transcription factor causes the increased repression of the production of the antagonistic transcription factor, tipping the balance towards the dominance of the first transcription factor. The double negative feedback loop reinforce this dynamic and the system remains in the same state, until an external stimulus disturbs it \autocite{FerrellJr:2002fh}.

Despite their simplicity, toggle switches can be powerful building blocks with which to create complex responses in a synthetic network. They have been used in isolation() or in tandem to create complex networks and signalling cascades. 

The toggle switches have been studied extensively and there are numerous studies based on a number of different methods of modelling and analysing the dynamics of this system. A summary can be found in Table~\ref{tab:refs}. Numerous studies have concluded that cooperativity is a necessary condition for bistability to arise (). \textcite{Lipshtat:2006wb} found that stochastic effects can give rise to bistability even without cooperativity in three kinds of switch: In the exclusive switch, in which there can only be one repressor bound at any one time, the switch in which there is degradation of bound repressors or the switch in which free repressor proteins can form a complex which renders them inactive as transcription factors \autocite{Lipshtat:2006wb}. In their study, \textcite{Ma:2012dt} found that the stochastic fluctuation in a system involving such a small number of molecules, like the toggle switch, uncovers effects that can not be predicted by the fully deterministic case. In their system, the toggle switch was found to be tristable, as small number effects render the third unstable steady state stable. In the study conducted by \textcite{Biancalani:2015vy}, they identified multiplicative noise as the source of bistability in the stochastic case. This bistability disappears of the noise source is reduced below a threshold, which in the toggle switch is represented by the repression strength. Thus if the repression strength falls below a certain threshold, the switch becomes monostable. In the genetic toggle switch the source of noise is the weakness of the repression so thus, increasing repression strength one decreases the source of noise and increases the stability of the switch \autocite{Warren:2005kea}. \textcite{Warren:2005kea} concluded that the exclusive switch is always more robust than the general switch, since the free energy barrier is higher. As is clear from above, there is yet to exist a consensus on the stability a switch is capable of, and the most appropriate method of modelling it. Different methods arrive at different conclusions, creating a confusion on which behaviour to be expected by the experimentalist for even a simple system like the toggle switch, consisting of just two genes. The toggle switch cannot be used as a building block of a bigger, more complex system, until its behaviour can be predicted otherwise designing such a system with predictable behaviour will be near impossible.


\begin{table}[]
\centering
\caption{Stability of switches found in literature}
\label{tab:refs}
\rotatebox{90}{
\begin{tabular}{ccccccc}
\toprule
                                        & \multicolumn{3}{c}{\textbf{Simple}}                                                                                                                                            & \multicolumn{3}{c}{\textbf{Double positive autoregulation}} \\ \midrule
                                        & \textit{Stability}                             & \textit{Reference} & \textit{Notes}                                                                                           & \textit{Stability}  & \textit{Reference}  & \textit{Notes}  \\  \midrule
\multirow{3}{*}{\textbf{Deterministic}} & Monostable                                     & Loinger 2007       & \begin{tabular}[c]{@{}c@{}}no cooperativity,\\ exclusive \& general\end{tabular}                         & Bistable            & Guantes 2008        &                 \\
                                        & \multirow{2}{*}{Bistable}                      & Gardner 2000       & copperativity \textgreater 2,                                                                            & Tristable           & Guantes 2008        &                 \\
                                        &                                                & Loinger 2007       & bound repressor degradation                                                                              & 4 steady steates    & Guantes 2008        &                 \\ \midrule
\multirow{7}{*}{\textbf{Stochastic}}    & Monostable                                     & Loinger 2007       & \begin{tabular}[c]{@{}c@{}}no cooperativity, \\ weak repression\end{tabular}                             & Tristable           & Lu 2014             &                 \\
                                        & \multirow{4}{*}{Bistable}                      & Lu 2014            &                                                                                                          &                     &                     &                 \\
                                        &                                                & Biancalani 2015    & \begin{tabular}[c]{@{}c@{}}exclusive,\\ controlled by noise strength\end{tabular}                        &                     &                     &                 \\
                                        &                                                & Lipshtat 2006      & no cooperativity                                                                                         &                     &                     &                 \\
                                        &                                                & Loinger 2007       & \begin{tabular}[c]{@{}c@{}}no cooperativity,\\ exclusive \& \\ bound repression degradation\end{tabular} &                     &                     &                 \\
                                        & \multicolumn{1}{l}{\multirow{2}{*}{Tristable}} & Loinger 2007       & \begin{tabular}[c]{@{}c@{}}no cooperativity,\\ strong repression\end{tabular}                            &                     &                     &                 \\
                                        & \multicolumn{1}{l}{}                           & Ma 2012            &                                                                                                          &                     &                     &                 \\ \bottomrule 
\end{tabular}
}
\end{table}

\clearpage



In order to solve this problem, we created a framework, StabilityFinder, that can be used to elucidate the stability each model is capable of, under conditions of parameter uncertainty. Using this framework one can study the different methods of modelling the switch and be able to attribute the differences in the results to each type of modelling. By using the same framework to compare the different modelling techniques we can get to the bottom of what the stability the switch is capable of and why each method produces a different result. We will use StabilityFinder to study three different models of the toggle switch. This methodology can also be used to uncover the design principles behind making a bistable switch, as well as those necessary to make a tristable switch. The methodology presented here can be used to study existing systems, toggle switches that exist in nature, as well as synthetic systems, if used as an aid to system design in synthetic biology. By identifying the parameter ranges that can give rise to the desired stability of a system, one can choose the parts of the genetic system accordingly. For example, if StabilityFinder dictates that gene expression must be low for a given stability, one can select a weak promoter or a low copy plasmid for their desired construct. 

% \autocite{Lipshtat:2006wb}
%To qualify as a switch the spontaneous switching rate must be much lower than the rates of the relevant processes in the cell.Numerous studies have concluded that cooperative binding is a necesasary condition for the emergence of the two distinct stable states (11,14)
%It was also observed that in the presence of cooperative binding stochastic effects contribute to the broadening of the parameter range which bistability occurs. 
%here show that stochastic effects enable bistability even without cooperative binding. Bistability takes place even when active proteins appear in high copy numbers.
%in the exclusive switch stochastic effects give rise to bistability even without cooperativity between transcription factors. Weak repression -> monostable, strong repression -> bistable. deadlock situation is impossible.
%bistability without cooperative binding: exclusive switch, degradation of bound repressors, free A and B proteins may form a complex which is not active as a transcription factor. 
%
% \autocite{Ma:2012dt}
%stochastic fluctuation is not negligible in such a small system
%a few simulational works having captured some new kinetic stability via KMC simulations including multi-stability. Experimentally and theoretically revealed an additional kinetic stable state nearby the traditionally regarded USS. The whole system could be from monostable to tristable under different conditions.  Source of additional kinetic stability: the discrete and fluctuate nature of molecule numbers, especially when they are often very close to or equal to zero.
%these are composite effects presented in small reaction systems that have pivotal molecules fluctuating on the verge of extinction.
%The results show that the behaviour of such a small system is quite different from what a fully deterministic model could predict. Besides fluctuation, there must be some features inherent in such a small system that contribute to the additional kinetic stability. it is the combined effects of the discrete, fluctuate, and non-negative features of molecular numbers in such a small system that result in the additional kinetic stability observed in experiments.
%
% \autocite{Lu:2014kc}
%Noise has been shown before to have a significant role in gene expression transitions and stability (9,10)
%Study effect of white Gaussian noise and shot noise on the dynamics of multi-state switches
%The two-way switch has two stable fixed points and a saddle point and the three-way switch has three stable fixed points and two saddle points.
%
% \autocite{Strasser:2012kt}
%protein variations of a differentiating cell influence the dynamics of the decision-making process and lead to stochastic transitions between the two steady states.
%probabilistic models of the toggle switch account for low copy numbers and intrinsic fluctuations
%contrary to deterministic models transitions between the two macroscopic regimes where one of the two genes dominates are possible due to the inherently noisy gene transcription even without cooperative binding of transcription factors.
%the inclusion of both mRNA and protein leads to an interesting change in system dynamics. the probabilistic two stage description exhibits complex multi-attractor dynamics without auto-activation and cooperativity. 
%assume monomeric transcription factor binding.
%Four attractors
%
% \autocite{Warren:2005kea}
%switch stability is decreased by phenomena that increase the noise in gene expression
%robustness against biochemical noise can be drastically enhances if switch is mutually exclusive
%Natural examples in lambda phage and human herpesvirus3. synthetic switches also constructed in vivo (9)
%switches usually flipped by external signals
%often so stable that they remain in one state until external trigger flips it. in lambda phage a spontaneous flip occurs as low as 10(to the)12 s (8). 
%region of bistability significantly larger for exclusive switch than that for general switch
%while the kinetic prefactor is toughly the same for both switches, the free energy barrier for flipping is significantly higher for the exclusive switch than for the general switch. 
%
% \autocite{}
%citation = Biancalani 
%Cooperative binding, more than a single TF can bind the DNA SEQUENCE AND THE BINDING PROBABILITY DEPENDS ON WHETHER there are molecules already bound to the sequence. (15) bimodality reported in a synthetic budding yeast system which concluded that the bimodal behaviour is induced by mRNA noise
%(11)(14) genetic toggle switch bimodal behaviour due to demographic noise even in the absence of cooperative binding.
%Here tehy show that bimodality is induced by multiplicative noise: the noise strength vanishes at the bimodal states whereas it is maximal at the single stable fixed point. Bimodal behaviour ceases to occur if the noise strength in the system is reduced below a critical threshold. In genetic toggle switches noise is controlled by the repression strength. 
%
% \autocite{}
%citation = Wang 2007
%
%A system is termed bistable if it can switch between two distinct stable steady states but cannot rest in intermediate states under the excitation of external stimuli. Bistable systems are building blocks of larger regulatory elements: geentic networks and signalling cascades.
%Noise exists extensively in biological systems with a small number of molecules due to the intrinsically stochastic nature of biochemical reactions involved or because of environmental fluctuations.
%Since living systems are usually optimized to function in the presence of stochastic fluctuations(23) the biochemical networks must withstand considerable variations and random perturbations of biochemical parameters (24)
%Here show that in the case of the toggle switch the multiplicative noises resulting from stochastic fluctuations in degradation rates can induce switching
%
% \autocite{}
%citation = Siegal-Gaskins 2015
%Sturn's theorem from algebraic geometry
%dimer-dimer toggle switch is bistbale over a greater range of parameters then the monomer-dimer switch. There is no single parameter more improtant than any other for achieving bistability.
%
% \autocite{Chen:2012bd}
%Some tremendous challeneges still remain, how to rationally build genetic devices with predetermined erformance. One of the most used genetic devices is the toggle switch (2,7,15)
%Bistability confers cells the ability to stochastically switch between phenotyipic states to generate diversity in a population (17,18) but also makes it a core component of gene circuits for higher-level sequential logic processes such as adaptive learning (2,7,15)
%Here provide method for rational design of genetic toggle switches with predetermined bistability. Single postivie autoregulation
%establishes equivalence between RBS DNA sequence and its bistability. 

%-%-%-%-%-%-%-%-%-%-%-%-%-%-%-%%-%-%-%-%-%-%-%-%-%-%-%-%-%-%-%%-%-%-%-%-%-%-%-%-%-%-%-%-%-%-%%-%-%-%-%-%-%-%-%-%-%-%-%-%-%-%
%Why stability checker is useful
%
% \autocite{Toni:2009tr}
%In biological systems we lack reliable information about parameters
%In ABC methods, the evaluation of the likelihood is replaced by a simulation-based procedure (Pritchard et al. 1999; Beaumont et al. 2002; Marjoram et al. 2003; Sisson et al. 2007).
%information about sensitivity of the model to parameter variation.
%
%Let q be a parameter vector to be estimated. Given the prior distribution p(q), the goal is to approximate the posterior distribution, p(qjx)ff(xjq)p(q), where f(xjq) is the likelihood of q given the data x. 
%
%In ABC SMC, a number of sampled parameter values (called particles), {q(1), ., q(N )}, sampled from the prior distribution p(q), are propagated through a sequence of intermediate distributions, p(qjd (x0, x)ei), iZ1, ., TK1, until it represents a sample from the target distribution p(qjd (x0, x)eT). The tolerances ei are chosen such that e1O/OeTR0, thus the distributions gradually evolve towards the target posterior.
%
%
% \autocite{Barnes:2011hh}
%the modeling and model evaluation/characterization is incorporated directly into the design stage. 
%The statistical nature of the method has many attractive features including the handling of stochastic systems, the ability to perform model selection and the handling of parameter uncer- tainty in a well-defined manner.
% We used the ABC-SysBio and cuda-sim softwares (19, 35), which takes as input a set of SBML files and as such can be used by bioengineers and experimentalists to rationally com- pare their competing designs for a system. 





