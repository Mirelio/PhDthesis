




Synthetic biology aims at using engineering principles for the construction of new biological systems. A parallel is drawn between the design and construction process in engineering and synthetic biology. Emphasis is put on modularity and standardization of the parts in play as well as on the separation of design and construction~\autocite{Agapakis:2009bt}. Ideally synthetic biology would have a toolkit of interchangeable parts that can be chosen for each application. Like nuts and bolts whose functions and features are well known and characterised, synthetic biologist aims to have an equivalent toolkit of fully characterised promoters and genes that can be selected to produce the system of choice. Nevertheless, biology does not conform with this idealised scenario. The cell is a noisy environment with a large number of unknowns and biological parts exhibit crosstalk, making the system unpredictable. This only highlights the need for the use of better computational tools for the understanding of a given biological system. Better tools are needed not only for the design of new synthetic systems but also for the better understanding of existing systems.     

Here I have addressed both of these issues by applying Bayesian statistics to synthetic biology problems. An existing package for Bayesian model selection was used, as well as two new computational tools built. The first tool that was constructed, StabilityFinder, is used for the design of synthetic systems and the second, ABC-Flow, is used for the inference of the parameters of existing biological systems. I applied the above tools to the understanding of a commonly found motif, the genetic toggle switch. The genetic toggle switch is an essential part of the synthetic biology toolkit and understanding its features is of great importance for the success of future applications of this motif. 

The first synthetic system design problem that was approached here was the design of a robust toggle switch. Bayesian model selection was used to determine that the addition of feedback loops to the classic design of the toggle switch can increase its parametric robustness and improve the system's ability to realise the predefined design objectives. This finding can be used for the construction of a reliable synthetic toggle switch, capable of moving this motif from the lab and onto a real word application.   

The first tool developed here, StabilityFinder uses Bayesian statistics to identify the parameter values that give rise to the desired stability for a given model and can be used to design novel synthetic switches. StabilityFinder was used to gain further insights into the stability the genetic toggle switch is capable of. It was shown that the genetic toggle switch is capable of multistable behaviour, and the design principles behind each behaviour were uncovered. This insight can be used to construct new synthetic switches that can behave in the desired way. The successful construction of switches with more than two possible steady states can extend the number of applications the switch can be used for. StabilityFinder was also used to study the genetic toggle switch under different modelling abstractions. It was shown that the \acrshort{qssa} cannot always be justified in the study of system behaviour. More general mass action models were used to study the switch model using deterministic and stochastic dynamics and it was found that it is capable of multistationarity. 

Robustness of the bistable toggle switch was examined using StabilityFinder and it was found that the addition of double positive feedback loops to the classical system increases the parametric robustness of the system. This result complements the first conclusion drawn in this thesis. Both methods look at the ability of this model to behave like a switch, but define the behaviour in different ways. Using ABC-SysBio a switch-like behaviour was defined as three design objectives that needed to be fulfilled, with a predefined time to reach each state and level of protein. The models were ranked for their ability to fit this very specific behaviour. StabilityFinder defined a switch-like behaviour as two clusters of steady state values of the two proteins in the system within a predefined time frame. It does not automatically rank the models under consideration, and the robustness analysis is applied after-the-fact. The two methods agree that the addition of positive feedback loops increases the parametric robustness of the switch, thus strengthening the argument.  

The toggle switch was also studied experimentally. The sensitivity of the switch to both inducer concentrations was studied. The switch was also observed switching states over time for both sides of the switch. The problem of parameter inference for flow cytometry data was addressed by developing ABC-Flow.  ABC-Flow is used to fit stochastic computational models to data obtained from flow cytometry. It was shown that it can be used to infer the parameter values that give rise to the observed experimental data collected here. A computational model of the toggle switch was fit to one and two dimensional data. This enables the parameterization of quantitative models using 2D flow cytometry data. This can provide further insights into the system under study that could not be otherwise obtained. The behavioural properties owed to different sub-parts can be untangled and further our understanding of the underlying effects at play in a given system.

The next step will be to move towards the testing of the predictions made here experimentally. The realisation of the switches with added positive feedback - using the construction strategy described here - would enable the testing of system robustness. Further, using the design principles of multistable switches predicted here, a switch with three or four states could be constructed experimentally. This will expand the toolkit of modules that can be used for synthetic biology applications. Another important step from here would be to move towards the integration of multiple devices. For synthetic biology to move into real clinical and industrial applications, the systems we can design and build will have to become more complex and reliable. This will require the successful interplay between multiple devices like the one studied here. In the future multiple switch modules can be combined to create more complex system behaviours. Switches can be combined to work in tandem with other kinds of modules like actuators and oscillators to perform complex functions in the cell. This will not be a trivial process due to retroactivity and crosstalk between devices~\autocite{DelVecchio:2008gy}.


In this thesis I studied the genetic toggle switch computationally and experimentally. I developed two computational tools that can be used for the study of genetic systems in systems and synthetic biology. I used them to uncover important aspects of the toggle switch system, a known regulatory motif in natural and synthetic systems. The work presented here advances our understanding of the design of novel switches as well as of an existing synthetic genetic toggle switch.



%Also, the construction of the novel synthetic switch designs will enable the experimental verification of system robustness. The new switches can be characterized using ABC-Flow to compare the parameter values that produce a bistable switch. This will allow the creation of a parts library that can be used for the switch, and each design can be used for the appropriate application.
%Flow cytometry is an experimental method used to detect the levels of fluorescence in individual cells.
%The parameter values necessary to achieve a desired stability can be used as a guide for the selection of appropriate parts when building a new system. 
%The genetic toggle switch can be used for a variety of applications as it can be used as a memory device while only
% The genetic toggle switch can be used as a memory device, as once it has flipped to one side, it will remain there until perturbed again and can be used for a number of useful applications in synthetic biology.
%The work carried out here could be extended in the future to real-world applications of the genetic toggle switch. One such application is in the field of microbiome engineering. The mico
