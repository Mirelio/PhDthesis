\section{Discussion}

Here we presented a methodology, StabilityChecker, which can identify the region of parameter space that can produce the stability of choice. We demonstrated its use on some known models and extracted stability and robustness information from them. We compared the stability profile of the Gardner toggle switch when modelled deterministically and stochastically in order to uncover the differences that arise from the addition of noise in the modelling. We found that the stochastic model showed increased robustness to noise.

We then applied StabilityChecker to the Lu switches in order to uncover the design principles that make a switch bistable versus tristable. We found that the gene expression parameters were critical for making a tristable switch bistable. This would be in agreement with the dynamics of a tristable switch, in which the third steady state occurs when there is a deadlock situation between the two proteins. When there are small numbers of both proteins involved, one represses the production of the other resulting in both promoters being repressed \autocite{Ma:2012dt}. Given our result we can extrapolate that a higher rate of protein production eliminates the possibility of this deadlock situation happening. Once a promoter is free to produce protein it will produce it in a fast enough rate so that that protein dominates the system and represses the antagonizing promoter before it has the chance to repress it. This dynamic would explain the fact that when all the priors remained the same but gene expression was increased by an order of magnitude, the tristable switch became bistable. 

We also applied StabilityChecker to a synthetic biology design problem. We used two models of the switch, one simple model consisting of two mutually repressive transcription factors and a model with added double positive auto-regulation. Comparing the two models, both capable of bistable behaviour, using StabilityChecker we found that the model with added double positive feedback loops is more robust to parameter fluctuations. This makes it a better candidate for building new synthetic devices based on the toggle switch design. We identified the parameter region within which this models are bistable, information that is important when building such a device in the lab. In the future, by selecting the system components accordingly, the parameter values can be adjusted \textit{in vivo}. For example, the parameter value corresponding to the translation initiation rate can be chosen by selecting the appropriate RBS sequence which given a nucleotide sequence will produce the desired rate \autocite{Holtz:2010bm}, a method developed by Salis \autocite{Salis:2009gk}. Another method to tweak the parameter values \textit{in vivo} is to select the promoter to have the strength corresponding to the levels of gene expression and repression desired. Activity of each promoter can be measured and standardised \autocite{Kelly:2009bj} making this process possible. For a system requiring more than one promoter, these can be efficiently selected from a promoter library using a genetic algorithm created by \textcite{Wu:2011bq}. These standardised interchangeable components with known sequence and activity are what synthetic biology classes as BioBricks \autocite{Kelly:2009bj,Canton:2008fv}. These can be selected and used to construct a desired system and replicate the parameter values found using StabilityChecker.

The methodology we presented here can be applied to a variety of problems as demonstrated. It can be applied to any problem of finding the parameter values that can produce a desired stability between two species. It can be used to design new systems of desired stability and help identify the appropriate parts to use by identifying the rates within which these parts need to operate. It can also be used to examine existing systems and give an insight on the underlying mechanisms that allow for the given stability to occur. 


