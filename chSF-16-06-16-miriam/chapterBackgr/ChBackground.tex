
\section{Introduction to synthetic biology}

Synthetic biology aims at the rational design and construction of biological parts, devices, and systems in order to engineer organisms to perform new tasks~\autocite{Lu:2009ez,Andrianantoandro:2006bi}. A part is a basic unit, like a promoter or a ribosome binding site that when combined with other parts will make a functional unit, a device~\autocite{Heinemann:2006ht}. A device processes inputs, performs functions and produces outputs~\autocite{Andrianantoandro:2006bi}. A system comprises of a collection of devices.     

Emphasis is put on the use of engineering principles such as modularity, standardisation, use of predictive models and the separation of design and construction~\autocite{Agapakis:2009bt, Heinemann:2006ht}. A hierarchy similar to computer science is used, with cells, pathways and biochemical reactions acting as computers, modules and gates respectively~\autocite{Andrianantoandro:2006bi}. 
       
Numerous applications of synthetic biology have emerged, from altering existing metabolisms to producing synthetic drugs ~\autocite{Holtz:2010bm} or creating new synthetic life forms ~\autocite{Agapakis:2009bt}. Despite the successes there is still a lack of predictive power due to the stochasticity and lack of complete knowledge of the cellular environment ~\autocite{Andrianantoandro:2006bi}.

Synthetic biology is now entering an age where simple synthetic circuits have been built, such as toggle switches \autocite{Gardner:2000vha, Kramer:2004kq, Isaacs:2003ht, Ham:2008hh, Deans:2007cya, Friedland:2009ce}, oscillators~\autocite{Stricker:2008jqa, Fung:2005jd, Tigges:2009jx} and pulse generators~\autocite{Basu:2004gn}, but larger circuits have proven more difficult \autocite{XXX}. The leap from building low-level circuits to assembling them into complex networks has yet to be made successfully~\autocite{Lu:2009ez}, and predictable circuit behaviour remains challenging \autocite{XXX}. Efforts to do so are plagued by intra-circuit crosstalk and incompatibility, as well as cellular noise, which can render synthetic networks non-functional \textit{in vivo} \autocite{XXX}. 

\section{System design in synthetic biology}

Creating synthetic devices that are robust to changing cellular contexts will be key to the success of synthetic biology. Unknown initial conditions and parameter values as well as the variability of the cellular environment, extracellular noise and crosstalk makes the majority of synthetic genetic devices non-functional~\autocite{Chen:2009ea}. Designing devices robust to this environment will lead to reliable behaviour of the systems.
When faced with a set of competing designs for a given genetic circuit, one is likely to choose the simplest possible model that can achieve the desired behaviour. However, simple systems are often the least robust. Feedback loops are well known key regulatory motifs ~\autocite{Brandman:2005ci}. Negative feedback loops are essential for homeostasis and buffering ~\autocite{Thomas:1995id} thus increasing robustness to extrinsic noise sources and positive feedback loops can generate multistationarity in a system ~\autocite{Thomas:1995id}. Incorporating this kind of additional feedback interactions can make a design more robust and reliable. 
Maximising production is an important goal for a metabolic engineering project if it is to produce an economically viable substance ~\autocite{Holtz:2010bm}. Network topologies and parameter values of different toggle switch designs are explored here in order to identify the design that maximises robustness and distance between steady states. This ensures the reliable production of the product with the greatest distance between the on and off states of the switch. 
 In the future, by selecting the system components accordingly, the parameter values can be adjusted \textit{in vivo}. For example, the parameter value corresponding to the translation initiation rate can be chosen by selecting the appropriate RBS sequence which given a nucleotide sequence will produce the desired rate ~\autocite{Holtz:2010bm}, a method developed by~\textcite{Salis:2009gk}~\autocite{Salis:2009gk}. Another method to tweak the parameter values \textit{in vivo} is to select the promoter to have the strength corresponding to the levels of gene expression and repression desired. Activity of each promoter can be measured and standardised~\autocite{Kelly:2009bj} making this process possible. For a system requiring more than one promoter, these can be efficiently selected from a promoter library using a genetic algorithm created by~\textcite{Wu:2011bq} ~\autocite{Wu:2011bq}. These standardised interchangeable components with known sequence and activity are what synthetic biology classes as BioBricks ~\autocite{Kelly:2009bj,Canton:2008fv}. These can be selected and used to construct a desired system and replicate the parameter values found in the scan presented here.

The first computational approach for the tuning of robust synthetic networks was that of~\textcite{Batt:2007jl}~\autocite{Batt:2007jl} where they examined the problem of finding a subset of the parameter set for which a given property was satisfied for all the parameters. ~\textcite{Chen:2009ea}~\autocite{Chen:2009ea} used the fuzzy  dynamic game method to solve the minimax regulation design problem of synthetic genetic networks. In that method the worst case effect of all disturbances is minimised for a given network. An evolutionary algorithm has also been used to solve the robust design problem by evolving the parameters of the system in order to make it more robust to cellular disturbances by~\textcite{Chen:2011hj}~\autocite{Chen:2011hj}. The added value of the methodology presented here is that the network structure in addition to the network parameters are adjusted to select a network that can robustly create the desired behaviour.	

\section{Introduction to Biochemical Modelling}
\subsection{Graphical representation of biochemical systems}

It is common to represent coupled biochemical reactions graphically. In a graph, as shown in Figure ~\ref{fig:Toggle switch example}, nodes represent the species and the edges represent an interaction between the species it connects, in which a transcription factor directly affects the transcription of a gene ~\cite{alon:2007b}. An arrow at the end of an arc represents activation, i.e. that when the transcription factor binds to the promoter the rate of transcription of the gene increases. A flat line perpendicular to the arc at the end of an arc represents repression, i.e. that when the transcription factor binds to the promoter the rate of transcription of the gene decreases ~\cite{alon:2007b}.

\subsection{Deterministic and Stochastic modelling}

Modelling attempts to describe the elements and dynamics of the biochemical system of interest. It is a tool used for integrating knowledge and experimental data as well as for making predictions about the behaviour of the system~\autocite{wilkinson:2006}. When modelling a biochemical system it is generally assumed that the rates of a reaction are directly proportional to the concentration of the reactants, raised to the power of their stoichiometry~\autocite{wilkinson:2006}. This is known as mass-action kinetics and is used in this work to model the various systems.  
There are two main ways of modelling a system, deterministically and stochastically. Deterministic modelling utilises ordinary differential equations (ODE) and models the concentrations of the species (proteins or other molecules) by time-dependent variables~\autocite{deJong:2002ft}. Rate equations are used to model gene regulation where the rate of production of a species is a function of the concentrations of the other species~\autocite{deJong:2002ft}. When modelling deterministically the model is viewed as a system which, with sufficient knowledge of the system, its behaviour is entirely predictable. Nevertheless we are still a long way away from having complete knowledge of a system of interesting size ~\cite{wilkinson:2006}. Deterministic modelling also assumes a homogenous mixture where species concentrations vary continuously and deterministically, assumptions that often are not met \textit{in vivo}. A cell is spatially and temporally separated, due to small molecule numbers and fluctuations in the timing of processes~\autocite{deJong:2002ft}.  
   
In stochastic modelling, species are measured in discrete amounts rather than concentrations and a joint probability distribution is used to express the probability that at time \textit{t} the cell contains a number of molecules of each species~\autocite{deJong:2002ft}. It takes uncertainty into account and does not assume a homogenous mix. It is thus often more appropriate for modelling cellular systems, although more computationally intensive. In stochastic systems the Gillespie algorithm is widely used to simulate the time-evolution of the state of the system~\autocite{wilkinson:2006}. The algorithm, developed by~\textcite{Gillespie:1977ww}~\autocite{Gillespie:1977ww} can be summarised in four steps:
\begin{enumerate}
\item Number of molecules in the system initialised
\item Two random numbers generated, one to determine which reaction will occur next and one to determine the time step
\item Time step increased and molecule counts updated according to Step 2 
\item Repeat from Step 2 until total simulation time reached
\end{enumerate}
   
 
\subsection{Steady state and stability}

In a steady state, the state of a system remains fixed. In non-linear systems, like the ones systems biology deals with, there is generally not an analytical solution thus the system has to be solved numerically. A stable steady state is defined as a fixed point whose nearby points approach the fixed point~\autocite{kaplan:1959}. This means that after a small perturbation the system will quickly return to the steady state. An unstable steady state is one which if the system is perturbed slightly then it moves away from the steady state~\autocite{konopka:2007}.    
 
 
\section{The genetic toggle switch}

One of the most common devices used in synthetic biology is the genetic toggle switch. A toggle switch consists of a set of transcription factors that mutually repress each other~\autocite{Gardner:2000vha}. Genetic switches play a major role in binary cell fate decisions like stem cell differentiation, as they are capable of exhibiting bistable behaviour. Bistability of a system is defined by the existence of two distinct phenotypic states but no intermediate state. Bistability is a property that is important in nature and a valuable resource to tap into in synthetic biology. It allows cells to alter their response to environmental cues and increases the overall population fitness by 'hedge-betting' the response of the population \autocite{XXX}. 

\subsection{Importance in natural systems}
In developmental processes, bistability ensures that the differentiating cell will follow one pathway, or the other, with no possible intermediate phenotypes. This is vital for the correct development of a cell in a specific pathway. One example is the trophectoderm differentiation pathway, in which a mutually inhibitory toggle switch exists between Oct3/4 and Cdx2. This determines whether an Embryonic Stem cell will differentiate into a Trophectoderm cell, if Cdx2 dominates the system, or an Inner Cell Mass cell if Oct3/4 dominates~\autocite{Niwa:2005fz}. Bistability is critical in this system as a cell must differentiate into either a trophectoderm cell or an inner cell mass cell, thus the signal to do so must be straightforward. In the case of the GATA1 and PU.1 toggle switch, the transcription factor pair controls the fate of the common myeloid progenitors, and the two possible differentiation paths are erythroid and myeloid blood cells~\autocite{Chickarmane:2009by}. The double-negative feedback loop created by the mutually repressive pair of transcription factors sustains the system in balance until an external stimulus causes one of the two transcription factors to increase in concentration. The increased concentration of one transcription factor causes the increased repression of the production of the antagonistic transcription factor, tipping the balance towards the dominance of the first transcription factor. The double negative feedback loop reinforces this dynamic and the system remains in the same state, until an external stimulus disturbs it~\autocite{FerrellJr:2002fh}.

\subsection{Uses in synthetic biology}
Despite their simplicity, toggle switches can be powerful building blocks with which to create complex responses in a synthetic network. They can be used in isolation or in tandem to create complex networks and signalling cascades. The toggle switch has been used for the regulation of mammalian gene expression~\autocite{Deans:2007cya, Kramer:2004kq}. Other synthetic applications of the toggle switch include the construction of a synthetic genetic clock~\autocite{Atkinson:2003tu}, of a predictable genetic timer~\autocite{Ellis:2009hka}, and the formation of biofilms in response to engineered stimuli~\autocite{Kobayashi:2004cv}. These applications are modifications of the classical toggle switch~\autocite{Gardner:2000vha}, and to our knowledge no application made of a cascade or collection of the switch has been successful. This would make more complex applications possible and could be used to solve real-life problems. For example, an analog-to-digital converter to translate external stimuli like the concentration of an inducer into an internal digital response, or programmable bacteria to move from point to point up different chemical gradients~\autocite{Lu:2009ez}. For a review on current circuits see~\autocite{Khalil:2010hm} and for possible future applications see~\autocite{Lu:2009ez}. This leap will be difficult to achieve before first being able to build robust and well characterised individual switches.

\subsection{Modelling the genetic toggle switch} 
The toggle switch motif has been studied extensively and there are numerous studies based on a number of different methods of modelling and analysis of the dynamics, including both deterministic and stochastic approaches. Deterministic modelling utilises ordinary differential equations (ODE) and models the concentrations of the species (proteins or other molecules) by time-dependent variables~\autocite{deJong:2002ft}. When modelling deterministically the model is viewed as a system whose behaviour is entirely predictable, given sufficient knowledge. In stochastic modelling, species are measured in discrete amounts rather than concentrations and a joint probability distribution is used to express the probability that at time \textit{t} the cell contains a number of molecules of each species~\autocite{deJong:2002ft,Wilkinson:2006td}. It takes uncertainty into account and is thus often more appropriate for modelling cellular systems, although more computationally expensive. In stochastic systems the Gillespie algorithm is widely used to simulate the time-evolution of the state of the system~\autocite{Warren:2005kea}.
\par
The conclusions drawn about the stability and robustness of the toggle switch also vary between the different modelling approaches. Numerous studies have concluded that cooperativity is a necessary condition for bistability to arise~\autocite{Gardner:2000vha, Walczak:2005ds, Warren:2004baa, Warren:2005kea, Cherry:2000wi}. However, ~\textcite{Lipshtat:2006wb} found that stochastic effects can give rise to bistability even without cooperativity in three kinds of switch; the exclusive switch, in which there can only be one repressor bound at any one time, a switch in which there is degradation of bound repressors, and the switch in which free repressor proteins can form a complex, which renders them inactive as transcription factors~\autocite{Lipshtat:2006wb}. In another study, \textcite{Ma:2012dt} found that the stochastic fluctuations in a system involving such a small number of molecules, like the toggle switch, uncovers effects that can not be predicted by the fully deterministic case~\autocite{Ma:2012dt}. In their system, the toggle switch was found to be tristable, as small number effects render the third unstable steady state stable. \textcite{Biancalani:2015vya} identified multiplicative noise as the source of bistability in the stochastic case~\autocite{Biancalani:2015vya}. ~\textcite{Warren:2005kea} concluded that the exclusive switch is always more robust than the general switch, since the free energy barrier is higher~\autocite{Warren:2005kea}. A summary of the toggle switch models is shown in Table~\ref{tab:refs}. As is clear from above, there is yet to exist a consensus on the stability a switch is capable of, and the most appropriate method of modelling it. Different methods arrive at different conclusions, creating confusion on which behaviour to be expected by the experimentalist for even a simple system like the toggle switch, consisting of just two genes. The toggle switch cannot be used as a building block of larger, more complex systems until its behaviour can be predicted accurately. Until then, designing systems with predictable behaviour will be near impossible.


\begin{table}[]
\centering
\caption{Summary of stability for the toggle switch found via different modelling approaches}
\label{tab:refs}
\rotatebox{90}{
\begin{tabular}{@{}ccccccc@{}}
\toprule
                                        & \multicolumn{3}{c}{\textbf{Simple}}                                                                                                                                            & \multicolumn{3}{c}{\textbf{Double positive autoregulation}} \\ \midrule
                                        & \textit{Stability}                             & \textit{Reference} & \textit{Notes}                                                                                           & \textit{Stability}  & \textit{Reference}  & \textit{Notes}  \\  \midrule
\multirow{3}{*}{\textbf{Deterministic}} & Monostable                                     & \autocite{Loinger:2007vma}       & \begin{tabular}[c]{@{}c@{}}no cooperativity,\\ exclusive \& general\end{tabular}                         & Bistable            & \autocite{Guantes:2008gs}        &                 \\
                                        & \multirow{2}{*}{Bistable}                      & \autocite{Gardner:2000vha}       & copperativity \textgreater 2,                                                                            & Tristable           & \autocite{Guantes:2008gs}        &                 \\
                                        &                                                & \autocite{Loinger:2007vma}       & bound repressor degradation                                                                              & 4 steady steates    & \autocite{Guantes:2008gs}        &                 \\ \midrule
\multirow{7}{*}{\textbf{Stochastic}}    & Monostable                                     & \autocite{Loinger:2007vma}       & \begin{tabular}[c]{@{}c@{}}no cooperativity, \\ weak repression\end{tabular}                             & Tristable           & \autocite{Lu:2014kc}             &                 \\
                                        & \multirow{4}{*}{Bistable}                      & \autocite{Lu:2014kc}            &                                                                                                          &                     &                     &                 \\
                                        &                                                & \autocite{Biancalani:2015vya}    & \begin{tabular}[c]{@{}c@{}}exclusive,\\ controlled by noise strength\end{tabular}                        &                     &                     &                 \\
                                        &                                                & \autocite{Lipshtat:2006wb}      & no cooperativity                                                                                         &                     &                     &                 \\
                                        &                                                & \autocite{Loinger:2007vma}       & \begin{tabular}[c]{@{}c@{}}no cooperativity,\\ exclusive \& \\ bound repression degradation\end{tabular} &                     &                     &                 \\
                                        & \multicolumn{1}{l}{\multirow{2}{*}{Tristable}} & \autocite{Loinger:2007vma}       & \begin{tabular}[c]{@{}c@{}}no cooperativity,\\ strong repression\end{tabular}                            &                     &                     &                 \\
                                        & \multicolumn{1}{l}{}                           & \autocite{Ma:2012dt}           &                                                                                                          &                     &                     &                 \\ \bottomrule 
\end{tabular}
}
\end{table} 
\clearpage



\section{Introduction to Bayesian statistics}

\begin{align*}
p(\theta|x) &= \frac{p(x|\theta)p(\theta)} {\displaystyle \int p(x|\theta)p(\theta)d\theta\frac{p(x|\theta)p(\theta)}{p(x)}}
\end{align*}

because
\begin{align*}
p(x)p(\theta|x) &= p(\theta)p(x|\theta)
\end{align*}

where $p(x|\theta)$ is the likelihood, $p(\theta)$ is the prior, and $\displaystyle \int p(x|\theta)p(\theta)d\theta$ is the evidence. This is the normalisation. 

Bayes factor: 
\begin{align*}
B_{12} = \frac{\displaystyle \int p(x|\theta, M_1)p(\theta, M_1)d\theta}{\displaystyle \int p(x|\theta, M_2)p(\theta, M_2)d\theta}
\end{align*}


In our case, O is the objective, and D is the design. Therefore:

\begin{align*}
p(O|D_1) = \int p(O|\theta,D_1)p(\theta|D_1)d\theta,
\end{align*}



This is the robustness, or evidence or marginal likelihood

\begin{align*}
p(O|D_1) &= \displaystyle \int p(O|\theta,D_1)p(\theta|D_1)d\theta, \\
p(O|D_1) &= \displaystyle \iiint_{\underline{\Theta}} p(O|\underline{\Theta})p(\underline{\Theta}|D_1)d\underline{\Theta}
\end{align*}

where $\underline{\Theta} = \{ \theta_1, \theta_2,\theta_3 \}$ 

Assuming the prior is uniform, and $a=0$:

\begin{align*}
p(O|D_1) &= \displaystyle \iiint_{\underline{\Theta}} p(O|\underline{\Theta})\frac{1}{b_1}\frac{1}{b_2}\frac{1}{b_3}d\underline{\Theta} \\
p(O|D_1) &= \frac{1}{b_1}\frac{1}{b_2}\frac{1}{b_3} \displaystyle \iiint_{\underline{\Theta}}p(O|\underline{\Theta})d\underline{\Theta}
\end{align*}


Assuming uniform likelihood:
\begin{align*}
p(O|D_1) &= \frac{1}{b_1}\frac{1}{b_2}\frac{1}{b_3} \displaystyle \iiint_{\underline{\Theta}_F}1d\theta_1\theta_2\theta_3+\frac{1}{b_1}\frac{1}{b_2}\frac{1}{b_3} \displaystyle \iiint_{\underline{\Theta}_F}Od\underline{\Theta}
\end{align*}


\subsection{Bayes' theorem}
\subsection{Bayesian inference}
\subsection{Model checking}
\subsection{Model comparison}
\subsection{Prior selection}





